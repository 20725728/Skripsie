\chapter{Safety Procedures}
The Stellenbosch Engineering laboratories were not used in this project however, once the prototype was developed there were several tests with the system on a dam. There are risks involved in this testing. Table \ref{saf} is the activity based risk assessment that shows the risks associated with the project and the mitigation taken to limit the effects of these risks. Overall the mitigations worked as there was risk occurrence. An incorrect connection was made leading to a short circuit and damaging the one ESC. 

\begin{table}
	\caption{Activity Based Risk Assessment}
	\label{saf}
\begin{tabular}{|m{2.5cm}|m{2.5cm}|m{1cm}|m{3cm}|m{6cm}|}
	\cl{1-5}
	\textbf{Activity} & \textbf{Risk} & \textbf{Risk Type} & \textbf{Classification of risk severety} &\textbf{ Mitigating Steps} \\
	\cl{1-5}
	\multirow{2}{2cm}{Launching Vessel} & Vessel sinking & E & Acceptable Risk & Ensure that the bungs are securely sealed \\
	\cl{2-5}
	& Vessel floating away & E & Acceptable Risk & Ensure a secondary rope is attached to the vessel to pull it back once it is off the trailer \\
	\cl{1-5}
	Connecting Electronic Equipment & Short Circuit & E & Substantial Risk & Double check which connection is the correct one before connecting them. \\
	\cl{1-5}
	Connecting Battery System & Electrical Shock & P & Possible Risk & Ensure the power switch is off before connecting the terminals and connect one terminal at a time and next touch both terminals. \\
	\cl{1-5}
	Operating Vessel & Cut limbs on thruster blades & P & Substantial Risk & Turn off the main power to the thruster before approaching the thrusters. \\
	\cl{2-5}
	& Electrical short & E & Substantial Risk & All connections are waterproofed to the best of abilities. \\
	\cl{2-5}
	& Vessel Collision & P\textbackslash{}E & Acceptable Risk & Be aware of surrounding and be ready to switch to manual control when operating the vessel. \\
	\cl{1-5}
	Moving around the vessel & Falling overboard & P & High Risk & Always wear a PFD while operating the vessel and keep a hold of the edges while moving around the vessel. \\
	\cl{2-5}
	& Capsizing vessel & P\textbackslash{}E & Possible Risk & Always wear a PFD while on the vessel and do not overload the vessel weight limit. \\
	\cl{1-5}
	Transporting vessel & Vehicle collision & P\textbackslash{}E & High Risk & Only a licensed driver may drive and must remain aware of the road conditions at all times. \\
	\cl{2-5}
	& Vessel trailer unhitching & P\textbackslash{}E & High Risk & Always ensure the vessel is securely hitched and check this routinely on long journeys. \\
	\cl{2-5}
	& Vessel falling off the trailer & P\textbackslash{}E & Substantial Risk & Always ensure the vessel is securely tied down when transporting. \\
	\cl{2-5}
	& Equipment falling off & E & Possible Risk & Secure the equipment in the vehicle and not on th vessel for transport. \\
	\cl{1-5}
\end{tabular}
\end{table}