\chapter{Testing and Results}
In order to test the system a series of tests were conducted to test each system individually before the overall system was tested. The individual systems tested are the GPS and compass, the PWM output, throttle, and the individual thrusters. Finally the full system was tested on the Stellenbosch canoe dam. 
\section{GPS and Compass}
The GPS and compass were tested on land. A baseline was established using a tracking application on a cellphone. The GPS and compass together with the cellphone were placed in a bag and a route was walked. The tracking data from both sources can then be downloaded and formatted as a Keyhole Markup Language (.kml) file and displayed on a map or plotted as has been done in figure \ref{fig:4:GPSMap}.\par
\begin{figure}
	\begin{center}
		\includegraphics[width=0.8\linewidth]{figures/graphGPSMap.jpg}
		\caption{The path used to compare the system GPS and the baseline GPS logger.}
		\label{fig:4:GPSMap}
	\end{center}
\end{figure}
In order to have a bearing to compare the compass reading with, the last two GPS points were used to calculate the bearin on which the vessel is pointing. Figure \ref{fig:4:bearingTest} shows the calculated bearing and the GPS bearing plotted together.\par
\begin{figure}
	\begin{center}
		\includegraphics[width=0.8\linewidth]{figures/graphBearingManual.jpg}
		\caption{The calculated bearing and compass bearing overlay.}
		\label{fig:4:bearingTest}
	\end{center}
\end{figure}
As can be seen in figure \ref{fig:4:GPSMap}, the two sets of GPS data match each other very closely with only a few areas where the systems GPS moves away from the baseline GPS points. In order to quantify the accuracy of the GPS, the distance between the coinciding points of the baseline GPS and the system GPS was calculated and plotted on box and whisker chart of figure \ref{fig:4:GpsBxWh}. As you can see the GPS is accurate to within \SI{5}{\meter} with only a few outliers, making up less than \SI{4}{\percent} of the data, falling slightly outside the \SI{5}{\meter} mark.\par
Similarly it can be seen on figure \ref{fig:4:bearingTest} that the systems compass bearing is closely following the calculated baseline bearing. Figure \ref{fig:4:bearingBxWh} shows a box and whisker plot of the error between the two bearings, with most of the data falling within a \SI{20}{\degree} range. There are more outliers than in the GPS test, however this is due to calculated bearing having several spikes. These spikes occur when the because the bearing is being calculated between GPS points that are often within 1m of each other and some accuracy can be lost. The noticeable spike of the system compass in figure \ref{fig:4:bearingTest} is not an error but caused by the bearing going around passed \SI{360}{\degree} to \SI{0}{\degree}.
 \begin{figure}
 	\begin{center}
 		\begin{subfigure}{0.45\linewidth}
 			\includegraphics[width = \linewidth]{figures/distBxWh.jpg}
 			\caption{Distance.}
 			\label{fig:4:GpsBxWh}	
 		\end{subfigure}
 		\begin{subfigure}{0.475\linewidth}
 			\includegraphics[width = \linewidth]{figures/bearingBxWh.jpg}
 			\caption{Bearing}
 			\label{fig:4:bearingBxWh}	
 		\end{subfigure}
 	\caption{Box and Whisker Graphs Of Distance and Bearing Between the System and the Baseline.}
 	\end{center}
 \end{figure} 
\section{PWM Output}
In order to adjust the speed of the thrusters, the PWM output must vary based on the throttle input. The PWM signal can easily be measured using an oscilloscope, therefore the system was set-up as shown in figure \ref{fig:4:PWMTest}. The throttle was connected to microcontrollers analogue inputs as it would be for the full system. The PWM outputs where then attached to the oscilloscope probes and the Arduino native programming port was used to power the system from a laptop. Each PWM signal was first tested individually before finally, both signals were tested simultaneously, each one on its own channel. \par
\begin{figure}
	\begin{center}
		\includegraphics[width=0.6\linewidth]{figures/PWMtest.jpg}
		\caption{Wiring diagram for the PWM oscilloscope test.}
		\label{fig:4:PWMTest}
	\end{center}
\end{figure}
As the throttle is adjusted the PWM signal should vary. There are three distinct positions, full forward, neutral and full reverse. that can be noted in table \ref{tab:3:PWM}. Theses three positions were tested to ensure that the range of the throttle was correctly calibrated. Finally the PWM signal was observed while slowly adjusting the throttle to ensure that the PWM signal varied linearly and in with a timely response. The results of the oscilloscope test are shown in figure \ref{fig:4:OscTest}. \par
It can be seen that the PWM responded well to the throttle inputs and met all three of the notable positions perfectly. Figure \ref{fig:4:OscTest:both} also shows that the two PWM outputs can have independent values to control each thruster as required. It should be noted that the images in figure \ref{fig:4:OscTest} all show a amplitude of \SI{3.3}{\volt} as the output was measured directly off of the Arduino microcontroller and before the signal was amplified using the logic level converter. It was confirmed that the converter accurately amplified the signal while maintaining the frequency and duty cycle of the signal.
\begin{figure}
	\begin{center}
		\begin{subfigure}{0.47\linewidth}
			\includegraphics[width = \linewidth]{figures/PWMfoward.jpg}
			\caption{Full Forward}
			\label{fig:4:OscTest:forward}	
		\end{subfigure}
		\begin{subfigure}{0.47\linewidth}
			\includegraphics[width = \linewidth]{figures/PWMneutral.jpg}
			\caption{Neutral}
			\label{fig:4:OscTest:neutral}	
		\end{subfigure}
		\begin{subfigure}{0.47\linewidth}
			\includegraphics[width = \linewidth]{figures/PWMreverse.jpg}
			\caption{Full Reverse}
			\label{fig:4:OscTest:reverse}	
		\end{subfigure}
		\begin{subfigure}{0.47\linewidth}
			\includegraphics[width = \linewidth]{figures/PWMboth.jpg}
			\caption{Both Outputs}
			\label{fig:4:OscTest:both}	
		\end{subfigure}
		\caption{Results of the PWM output displayed on an oscilloscope.}
		\label{fig:4:OscTest}
	\end{center}
\end{figure} 
\section{Throttle}
\begin{figure}
	\begin{center}
		\includegraphics[width = 0.8\linewidth]{figures/graphThrottle.jpg}
		\caption{The digital response of the throttle.}
		\label{fig:4:Throttleresponse}
	\end{center}
\end{figure}
In order to test the response of the throttle and to show that the linear potentiometers have good accuracy the graph in \ref{fig:4:Throttleresponse} was plotted. It can clearly be seen that the potentiometers are meeting the upper and lower thresholds listed in \ref{tab:3:POT} and can move independently of each other without interference. Figure \ref{fig:4:Throttleresponse} also shows that in the neutral position there is some variation caused by user input when moving the other throttle. As one pushes on the one throttle one, will slightly pull on the other without realizing. Furthermore, even if it looks like the throttle is back in the neutral position, even a small amount off from the original position results in a different value. This is why the neutral buffer in \ref{tab:3:POT} was applied and it can be seen that buffer covers the slight variations. 
\begin{figure}
	\begin{center}
		\includegraphics[width = 0.45\linewidth]{figures/thrusterBucket.jpg}
		\caption{The thruster in a bucket of water to test it without 'dry running'.}
		\label{fig:4:ThrusterTest}
	\end{center}
\end{figure}
\section{Thrusters}
The thrusters are designed to be operated in water as the water flow across the ESC is designed to dissipate the heat generated and prevent the ESC from overheating. Furthermore, the thruster is designed to push water and not air which have very different densities, so for these reasons the thrusters should not be 'dry run', and should only be operated while submerged. Figure \ref{fig:4:ThrusterTest} shows how, during the design and testing of the thrusters, a temporary water source was provide with a large bucket wherein the thrusters could be submerged to ensure that the thruster is responding to the PWM signal and giving the desired results.

\section{USV}
The final testing was conducted on the entire system as a whole by launching the vessel and operating it as a complete system. This section will first detail the procedure for setting up the system for a test followed by the details of how each test was carried out. Finally, the results of the full system tests will be discussed. 
\subsection{Set-up Procedure} 
The full system tests were carried out on the Stellenbosch canoe dam as it is a large enough body of water to sufficiently test the vessel and it is just outside of Stellenbosch so it is easily accessible. The system is transported completely disassembled so as to avoid any possible damages that could occur during transport. Therefore, before testing the vessel as well as the rest of the components must be set-up and checked before the vessel can be launched and the test can begin. Firstly the vessel and all safety equipment must be checked and prepared by following the checklist in table \ref{tab:4:boatCheck}. Once the vessel has be prepared the system can be assembled by following the checklist in table \ref{tab:4:equipCheck}. Once both of these checklists have been followed a final cursory sweep should be conducted to ensure that nothing has been missed and the vessel can be launched. Once the vessel is off of the trailer and in the water, the thrusters can be dropped down into their operational position. This will conclude the set-up of the vessel and the test can begin. 
\begin{table}
	\begin{center}
		
		\caption{Procedure to set-up the vessel and safety equipment}
		\label{tab:4:boatCheck}
		\begin{tabular}{|l|l|}
			\hline
			To Check: & Checked \\
			\hline
			All bungs are in place and secured. &\\
			\hline
			All tie downs are removed and stowed. &\\
			\hline
			The personal floatation device is aboard. &\\
			\hline
			The electrical fire extinguisher is aboard. &\\
			\hline
			There is no physical damage to the vessel that could cause leaks.&\\
			\hline
		\end{tabular}
	\end{center}
\end{table}
 \begin{table}
 	\begin{center}
 		\caption{Procedure to set-up the equipment and control system}
 		\label{tab:4:equipCheck}
 		\begin{tabular}{|p{0.65\linewidth}|l|}
			\hline
 			To Check: & Checked \\
 			\hline
 			Mount the throttle plate and secure the attaching bolts.&\\
 			\hline
 			Mount the control system under the throttle plate and secure the GPS and Compass onto the nose.&\\
 			\hline
 			Ensure that the thrusters are in the upright position and mount them onto the transom.&\\
 			\hline
 			Place the batteries at the back of the vessel, just in front of the transom.&\\
 			\hline
 			Ensure that the cut-off switch is in the off position and connect the thrusters to the batteries.&\\
 			\hline
 			Connect the control system to the thrusters. &\\
 			\hline
 			Check all connections to see that they are secure and that there are no open wires. &\\
 			\hline
 			Close the battery box and secure the battery box. &\\
 			\hline
 			Turn on the cut-off switch to power the system and check that the control system powers on. The thrusters should beep twice signifying that they are connected and receiving a signal.&\\
 			\hline
 			Wait until the GPS indicator light is on, signifying that the GPS has an valid GPS fix. &\\
 			\hline
 		\end{tabular}
 	\end{center}
 \end{table}
\subsection{Testing}
The testing occurred in two phases, the manual test and the autonomous test. The manual test was carried out by using the throttle to manually control the vessel and ensure that the thrusters are responding correctly and that the system is fully operational. The GPS points that the boat would navigate to under autonomous control were chosen while doing the manual test to ensure that the vessel would not navigate into any of the obstacles on the dam such as buoys and pumps.\par
Finally, for the autonomous test, the vessel was driven back to the starting position and the system was reset and autonomous navigation was selected. The vessel was still manned so that manual control could be taken at any point if it looked as though the vessel would collide with any obstacles. Once the vessel had navigated to its final point, manual control was selected to return the vessel to shore. 
\subsection{Results}
Finally, after all the testing had been completed, the results could be analysed and assessed to determine the validity and performance of the control system. The results will be discussed with the aid of graphs draw from the data collected during the tests. The performance of the control system will be determined by how accurately it stays on the course to the target location and the validity of the system will be determined by if the vessel reached the target points. \par
Looking at figure \ref{grph:4:distance}, which shows the distance to the target point over time, it can be seen that the vessel always gets to within the allowable distance of the target location. The system is working and navigates to all the given locations.\par
However, having a valid system is only useful if the system is accurately navigating to these points and not simply randomly lucking onto the target. To determine the performance of the system we can look at figures \ref{grph:4:BearingError} and \ref{grph:4:2bearings}. \ref{grph:4:BearingError} shows the error or difference between the vessels bearing and the bearing it should be following to reach the target. In a perfect system, this graph should be a flat line along the x-axis with spikes after the point has been reached and the bearing to target has been reached. Figure \ref{grph:4:2bearings} would similarly, show the vessel bearing following the target bearing exactly with the only deviation being as a point reached where the target bearing drastically changes and the vessel bearing must work its way back to the target bearing.\par 
Looking at the area between point 1 and 2 on \ref{grph:4:BearingError}, it can be seen that the graph is oscillating significantly around the x-axis with a heavier bias to the negative. This is not indicative of good performance, however when combining this with the plotted course in \ref{graph:4:Map} it can be seen that the vessel is making lots of small corrections but maintaining a relatively straight course to the target point.\par
Staying with \ref{graph:4:Map}, it is clear that after point 2, it looses performance as it begins to do large unnecessary loops to reach the target. This can also be seen in figure \ref{grph:4:distance} that the vessel begins to get closer until it is passed it and gets further away before looping around and once again getting closer and reaching the target. However, \ref{grph:4:BearingError} and \ref{grph:4:2bearings} show that the vessel bearing is still oscillating around the target bearing as between points 1 and 2. This is due to the compass loosing calibration and having 'dead spots'. The compass was not returning an accurate bearing and the system thought it was on the correct course. Once it had looped around and out of the compass 'dead spot' and it began returning the correct bearing, and could accurately navigate to the target location. 
\begin{figure}
	\begin{center}
		\begin{subfigure}{0.8\linewidth}
			\includegraphics[width = \linewidth]{figures/graphBearingError.jpg}
			\caption{Bearing Error.}
			\label{grph:4:BearingError}	
		\end{subfigure}
		\begin{subfigure}{0.8\linewidth}
			\includegraphics[width = \linewidth]{figures/graphDistance.jpg}
			\caption{Distance to Target}
			\label{grph:4:distance}	
		\end{subfigure}
		\begin{subfigure}{0.8\linewidth}
			\includegraphics[width = \linewidth]{figures/graphBearings.jpg}
			\caption{Bearings Comparison}
			\label{grph:4:2bearings}	
		\end{subfigure}
		\caption{Results of the PWM output displayed on an oscilloscope.}
		\label{fig:4:Results}
	\end{center}
\end{figure} 
\begin{figure}
	\begin{center}
		\includegraphics[width = 0.8\linewidth]{figures/graphMap.jpg}
		\caption{map}
		\label{graph:4:Map}
	\end{center}
\end{figure}