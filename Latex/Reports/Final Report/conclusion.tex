\chapter{Conclusion}
The objective of this project was to design and construct a prototype control system for an unmanned surface vessel. It would need to have a manual control option for testing in addition to the autonomous navigation system.
\section{Achieved}
The individual system test went well and no major issues were found. Although, the thrusters were slightly underpowered this was not a technical issue. The first autonomous test showed that the system could navigate to the points but due to the compass loosing calibration this was not accurate navigation. Finally, in the second autonomous test the system navigated accurately between the points and reached each target point. There was also a steady wind that blew the vessel off course and the system was able to counter this and still reach its target destination. The control system achieved its goal and would be able to be implemented into a larger system with minimal alterations. 
\section{Future Development}
A fully autonomous unmanned surface vessel would require more development in order for it to be safe and sustainable on the ocean. A primary requirement would be to add obstacle avoidance both above and below water so that the vessel could navigate debris, other vessels and shallow waters. This would also tie into a 'return home' feature that would allow the vessel to return to its starting point, avoiding any obstacles, if it encountered any problems or once it had completed its course. \par
It was mentioned that the thrusters in this project were underpowered and so it would be worthwhile investigating more powerful and efficient thrusters. However, more powerful thrusters would require a larger power supply. Future development can investigate renewable power regeneration such as solar to recharge the power supply and extend the range of the system. Furthermore, sensors to detect wind and current flow together with weather mapping could be used to plot the most efficient route the vessel can take working with the elements instead of against them. 