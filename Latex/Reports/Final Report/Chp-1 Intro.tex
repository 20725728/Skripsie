\chapter{Introduction}
\section{Background}
As technology has improved over the years, processes and systems have become more automated. Initially factories were replacing manual labour with automated machines but recently companies have been investigating self-driving cars and trucks. All over industries tasks are being automated or done remotely with fewer human involvement. \par
\vspace{0.6cm}
The ocean is the perfect area for unmanned surface vessels (USV) to be used as many of the issues faced with autonomous land vehicles such as self-driving cars are mitigated by open water. On the open water one gets a 360° of the surroundings of the vehicle and although there can still be high volumes of traffic in certain areas such as commercial shipping lanes, due to the expanse of the ocean these high traffic areas are avoidable. Finally, and probably the most desirable mitigating factor is that where a surface vehicle would need to look where the road surface is to follow it, an ocean vessel can move directly from point to point on any piece of water. \par
\vspace{0.6cm}
In South Africa there is a growing need to USVs with regards to ocean research and conservation. There has been a growing use of acoustic sensory systems to track dolphins and whales around the world. By combining this with the technology of USVs, a far larger area can be surveyed. \par
\section{Objectives}
This project will focus on the navigation and propulsion control of the USV. This is the building block of the USV upon which a future project can build by adding an obstacle avoidance system or renewable power sources to keep the USV operational for longer. This project will have the following objectives:
\begin{enumerate}
	\item Design and manufacture an electric surface vessel.
	\item Designing and manufacturing the control system that will give a pilot manual control over the electric surface vessel.
	\item Building on the manual control and implementing navigation control so that the electric surface vessel.
\end{enumerate}
\section{Motivation}
Currently the marine community is using these acoustic systems as stationary systems. By using the USV in conjunction with the USV the area that is studied can be greatly increased with fewer acoustic platforms as have been used in passed projects. Furthermore, the technology can be adapted for use in other industries such sonar surveying, defence and search and rescue. The use of USVs is becoming more prominent as a USV can be cheaper to operate and therefore organisations can either save costs in the case of sonar and acoustic research or in the case or marine patrols and search and rescue, USVs can be used to fill up the ranks of vessels and close the possible.\par
\vspace{0.6cm}
The tasks previously mentioned are often time consuming and the crew of the assigned vessel need time to rest whereas a fully autonomous USV can operate constantly, stopping only to replenish its energy source and with further developments such as solar charging, USVs could begin to operate indefinitely, having to only come in for services or if there is a problem with the system.\par
\vspace{0.6cm}
This report will first look at what is a USV and what equipment and knowledge is required to for a USV. A review of literature around these technologies to offer a background and explain the concepts. Following on, the designed system is described before going into more detail on each component both hardware and electronics and the software and algorithms used in the system. Finally the report will describe how each system is tested and the results will be discussed before a conclusion on the system can be made.
