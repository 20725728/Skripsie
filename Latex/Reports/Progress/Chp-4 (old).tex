\chapter{Planning}
The system has progressed well and although there have been delays, the system is very close to an operational state. Most of the physical hardware is complete and ready and there are only a few small elements left to complete before testing and refinement can commence.
\section{PCB}
The PCB for the motherboard still needs to be printed. An earlier version of the PCB was made, however, there were design flaws and it was designed to be an interim Motherboard as it did not have space for the GPS module. The design has been redrawn as the final design and is ready to print. The delay in PCB printing is due to a lack of doubled sided PCB board in stock and as soon as it is restocked, the board can be printed. Finally, the soldering of the components is a short process and then the motherboard will be ready to implement.
\section{Energy Potential}
Based on the datasheets and the minimal running that the thruster has done, it does not look like a \SI{12}{\volt} system will have enough potential to provide sufficient thrust to move and control the vessel. However, there is plenty of capacity with the current cells.\par
\vspace{0.4cm}
The simplest way to increase the potential of the battery would be to increase the number of cells. However, as the BMS is designed for \SI{21}{\volt} and 4 cells. An increase in the number of cells would require a new BMS. This would be a very costly solution, not only because of the new BMS required but mostly due to the fact that the cells are very costly. These cells are particulary expensive as they have a large capacity. Therefore, a alternative solution is to step up the voltage.\par
\vspace{0.4cm}
By stepping up the voltage on the thruster side, it would naturally step up the current on the battery side. Given that it is a \SI{100}{\ampere} BMS, running at the maximum current draw continuously and only draining the battery to \SI{50}{\percent}, would offer a 2 hour run time. This would be plenty of time to test the system between charges.\par
\vspace{0.4cm}
Ideally, a DC to DC step up voltage converter would work best. However, the high current loads of this system prevent any sort of device being applicable. Instead a similar device will have to be designed. A preliminary design is to use a standard inverter to step the voltage up to \SI{230}{\volt} alternating current where it can be stepped back down using either a transformer or for variability in the design, a variac. Finally the alternating current can be transformed back to direct current using a high power rectifier bridge. 
\section{Unmanned Control}
Once the system is operational and the manual controls have been confirmed to function as designed, the unmanned control can be implemented. The unmanned control will be given a prescribed path to follow. This path will consist of coordinate points and the order in which these points should be reached. If there is room in the scope of the project, an optimization algoritm can be incorporated to determine what is the ideal order in which to access the points.
\section{Safety}
The system is not yet completely developed and so there is planned refinement in the system and the encapsulation of the electronic hardware. Primarily for the purposes of safety. The battery will need to be placed within a sealable box. Furthermore any high voltage equipment should be contained in a water resistant container. Even low voltage equipment should be contained to ensure that no equipment is possibly damaged and lost. Finally, a kill switch will be implemented to cut the power from the battery to the rest of the system so ensure that if there is any accident or error the power can immediately be cut.
\section{Testing}
The testing of the vessel will be done in a controlled eviroment on a enclosed body of water. During all testing and operation there must be a person either on the vessel or alongside in another vessel so that manual control can be assumed if need be. There will be a fire extiguisher suitable for electrical fires on hand at all testing.
